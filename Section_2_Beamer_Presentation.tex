% Options for packages loaded elsewhere
\PassOptionsToPackage{unicode}{hyperref}
\PassOptionsToPackage{hyphens}{url}
\PassOptionsToPackage{dvipsnames,svgnames,x11names}{xcolor}
%
\documentclass[
  ignorenonframetext,
]{beamer}
\usepackage{pgfpages}
\setbeamertemplate{caption}[numbered]
\setbeamertemplate{caption label separator}{: }
\setbeamercolor{caption name}{fg=normal text.fg}
\beamertemplatenavigationsymbolsempty
% Prevent slide breaks in the middle of a paragraph
\widowpenalties 1 10000
\raggedbottom
\setbeamertemplate{part page}{
  \centering
  \begin{beamercolorbox}[sep=16pt,center]{part title}
    \usebeamerfont{part title}\insertpart\par
  \end{beamercolorbox}
}
\setbeamertemplate{section page}{
  \centering
  \begin{beamercolorbox}[sep=12pt,center]{part title}
    \usebeamerfont{section title}\insertsection\par
  \end{beamercolorbox}
}
\setbeamertemplate{subsection page}{
  \centering
  \begin{beamercolorbox}[sep=8pt,center]{part title}
    \usebeamerfont{subsection title}\insertsubsection\par
  \end{beamercolorbox}
}
\AtBeginPart{
  \frame{\partpage}
}
\AtBeginSection{
  \ifbibliography
  \else
    \frame{\sectionpage}
  \fi
}
\AtBeginSubsection{
  \frame{\subsectionpage}
}
\usepackage{amsmath,amssymb}
\usepackage{iftex}
\ifPDFTeX
  \usepackage[T1]{fontenc}
  \usepackage[utf8]{inputenc}
  \usepackage{textcomp} % provide euro and other symbols
\else % if luatex or xetex
  \usepackage{unicode-math} % this also loads fontspec
  \defaultfontfeatures{Scale=MatchLowercase}
  \defaultfontfeatures[\rmfamily]{Ligatures=TeX,Scale=1}
\fi
\usepackage{lmodern}
\usetheme[]{CambridgeUS}
\ifPDFTeX\else
  % xetex/luatex font selection
\fi
% Use upquote if available, for straight quotes in verbatim environments
\IfFileExists{upquote.sty}{\usepackage{upquote}}{}
\IfFileExists{microtype.sty}{% use microtype if available
  \usepackage[]{microtype}
  \UseMicrotypeSet[protrusion]{basicmath} % disable protrusion for tt fonts
}{}
\makeatletter
\@ifundefined{KOMAClassName}{% if non-KOMA class
  \IfFileExists{parskip.sty}{%
    \usepackage{parskip}
  }{% else
    \setlength{\parindent}{0pt}
    \setlength{\parskip}{6pt plus 2pt minus 1pt}}
}{% if KOMA class
  \KOMAoptions{parskip=half}}
\makeatother
\usepackage{xcolor}
\newif\ifbibliography
\usepackage{color}
\usepackage{fancyvrb}
\newcommand{\VerbBar}{|}
\newcommand{\VERB}{\Verb[commandchars=\\\{\}]}
\DefineVerbatimEnvironment{Highlighting}{Verbatim}{commandchars=\\\{\}}
% Add ',fontsize=\small' for more characters per line
\usepackage{framed}
\definecolor{shadecolor}{RGB}{248,248,248}
\newenvironment{Shaded}{\begin{snugshade}}{\end{snugshade}}
\newcommand{\AlertTok}[1]{\textcolor[rgb]{0.94,0.16,0.16}{#1}}
\newcommand{\AnnotationTok}[1]{\textcolor[rgb]{0.56,0.35,0.01}{\textbf{\textit{#1}}}}
\newcommand{\AttributeTok}[1]{\textcolor[rgb]{0.13,0.29,0.53}{#1}}
\newcommand{\BaseNTok}[1]{\textcolor[rgb]{0.00,0.00,0.81}{#1}}
\newcommand{\BuiltInTok}[1]{#1}
\newcommand{\CharTok}[1]{\textcolor[rgb]{0.31,0.60,0.02}{#1}}
\newcommand{\CommentTok}[1]{\textcolor[rgb]{0.56,0.35,0.01}{\textit{#1}}}
\newcommand{\CommentVarTok}[1]{\textcolor[rgb]{0.56,0.35,0.01}{\textbf{\textit{#1}}}}
\newcommand{\ConstantTok}[1]{\textcolor[rgb]{0.56,0.35,0.01}{#1}}
\newcommand{\ControlFlowTok}[1]{\textcolor[rgb]{0.13,0.29,0.53}{\textbf{#1}}}
\newcommand{\DataTypeTok}[1]{\textcolor[rgb]{0.13,0.29,0.53}{#1}}
\newcommand{\DecValTok}[1]{\textcolor[rgb]{0.00,0.00,0.81}{#1}}
\newcommand{\DocumentationTok}[1]{\textcolor[rgb]{0.56,0.35,0.01}{\textbf{\textit{#1}}}}
\newcommand{\ErrorTok}[1]{\textcolor[rgb]{0.64,0.00,0.00}{\textbf{#1}}}
\newcommand{\ExtensionTok}[1]{#1}
\newcommand{\FloatTok}[1]{\textcolor[rgb]{0.00,0.00,0.81}{#1}}
\newcommand{\FunctionTok}[1]{\textcolor[rgb]{0.13,0.29,0.53}{\textbf{#1}}}
\newcommand{\ImportTok}[1]{#1}
\newcommand{\InformationTok}[1]{\textcolor[rgb]{0.56,0.35,0.01}{\textbf{\textit{#1}}}}
\newcommand{\KeywordTok}[1]{\textcolor[rgb]{0.13,0.29,0.53}{\textbf{#1}}}
\newcommand{\NormalTok}[1]{#1}
\newcommand{\OperatorTok}[1]{\textcolor[rgb]{0.81,0.36,0.00}{\textbf{#1}}}
\newcommand{\OtherTok}[1]{\textcolor[rgb]{0.56,0.35,0.01}{#1}}
\newcommand{\PreprocessorTok}[1]{\textcolor[rgb]{0.56,0.35,0.01}{\textit{#1}}}
\newcommand{\RegionMarkerTok}[1]{#1}
\newcommand{\SpecialCharTok}[1]{\textcolor[rgb]{0.81,0.36,0.00}{\textbf{#1}}}
\newcommand{\SpecialStringTok}[1]{\textcolor[rgb]{0.31,0.60,0.02}{#1}}
\newcommand{\StringTok}[1]{\textcolor[rgb]{0.31,0.60,0.02}{#1}}
\newcommand{\VariableTok}[1]{\textcolor[rgb]{0.00,0.00,0.00}{#1}}
\newcommand{\VerbatimStringTok}[1]{\textcolor[rgb]{0.31,0.60,0.02}{#1}}
\newcommand{\WarningTok}[1]{\textcolor[rgb]{0.56,0.35,0.01}{\textbf{\textit{#1}}}}
\setlength{\emergencystretch}{3em} % prevent overfull lines
\providecommand{\tightlist}{%
  \setlength{\itemsep}{0pt}\setlength{\parskip}{0pt}}
\setcounter{secnumdepth}{-\maxdimen} % remove section numbering
\ifLuaTeX
  \usepackage{selnolig}  % disable illegal ligatures
\fi
\IfFileExists{bookmark.sty}{\usepackage{bookmark}}{\usepackage{hyperref}}
\IfFileExists{xurl.sty}{\usepackage{xurl}}{} % add URL line breaks if available
\urlstyle{same}
\hypersetup{
  pdftitle={Workshop: Section 2 - Basic Structure of oTree Experiments (Models, Pages, Templates)},
  colorlinks=true,
  linkcolor={Maroon},
  filecolor={Maroon},
  citecolor={Blue},
  urlcolor={blue},
  pdfcreator={LaTeX via pandoc}}

\title{Workshop: Section 2 - Basic Structure of oTree Experiments
(Models, Pages, Templates)}
\author{Philipp Chapkovski\\
University of Bonn\\
\href{mailto:chapkovski@uni-bonn.de}{chapkovski@uni-bonn.de}}
\date{September 11th - 12th, 2023}

\begin{document}
\frame{\titlepage}

\begin{frame}[fragile]{Introduction to oTree's Architecture}
\phantomsection\label{introduction-to-otrees-architecture}
\begin{itemize}
\item
  \textbf{Main components of any oTree project:}

  \begin{itemize}
  \item
    Apps (+\texttt{settings})
  \item
    Models
  \item
    Pages
  \item
    Templates
  \end{itemize}
\end{itemize}
\end{frame}

\begin{frame}[fragile]{Models in oTree}
\phantomsection\label{models-in-otree}
\begin{itemize}
\tightlist
\item
  Use to define data to store at each level
\item
  There are five (nested) models:
\item
  Session:

  \begin{itemize}
  \tightlist
  \item
    Subsession
  \end{itemize}
\item
  Participant:

  \begin{itemize}
  \tightlist
  \item
    Player
  \end{itemize}
\item
  Group\\
\item
  You can define new \textbf{fields} at Subession, Group, and Player
  level
\item
  You can store the data in \texttt{vars} at Session and Participant
  levels
\end{itemize}
\end{frame}

\begin{frame}[fragile]{Overall oTree data structure}
\phantomsection\label{overall-otree-data-structure}
\scritpsize

\begin{itemize}
\item
  \textbf{Session}: Top-level container for the entire experiment.

  \begin{itemize}
  \tightlist
  \item
    Consists of participants.
  \item
    Contains a sequence of apps.
  \end{itemize}
\item
  \textbf{App}: A component of the session.

  \begin{itemize}
  \tightlist
  \item
    Can have multiple rounds.
  \end{itemize}
\item
  \textbf{Round}: A single iteration within an app.

  \begin{itemize}
  \tightlist
  \item
    Divides players into groups or one large group (if
    \texttt{players\_per\_group=None}).
  \item
    Includes all players in a subsession.
  \end{itemize}
\item
  \textbf{Group}: A subset of players within a round.
\item
  \textbf{Subsession}: A set of all players in the round.
\item
  \textbf{Player}: Individual subject in a round.

  \begin{itemize}
  \tightlist
  \item
    Each player has a corresponding \textbf{Participant}, which stays
    the same across all apps and rounds.
  \end{itemize}
\end{itemize}
\end{frame}

\begin{frame}[fragile]{Pages in oTree}
\phantomsection\label{pages-in-otree}
\begin{itemize}
\tightlist
\item
  A page (in z-Tree term, screen) to show consists of 3 elements:
\item
  Page class
\item
  position of the Page in the sequence of pages
  (\texttt{page\_sequence})
\item
  the html (text) to show in the \texttt{Page.html}
\end{itemize}
\end{frame}

\begin{frame}[fragile]{Templates in oTree}
\phantomsection\label{templates-in-otree}
\begin{itemize}
\tightlist
\item
  within html of the page you can use:
\item
  plain html
  (\texttt{\textless{}b\textgreater{}hello,\ I\ am\ bold!\textless{}/b\textgreater{}}
  will result in: \textbf{hello, I am bold!})
\item
  CSS styles (look at the \href{https://getbootstrap.com/}{Bootstrap
  docs} for details)
\item
  JavaScript
\item
  and oTree own (kinda) template language to render at the server side
  (retrieving data)
\end{itemize}

\begin{block}{Anatomy of an oTree Page}
\phantomsection\label{anatomy-of-an-otree-page}
\begin{itemize}
\item
  An oTree page consists of several built-in methods that control its
  behavior and appearance. \scriptsize \#\#\#\# Main Built-In Methods:
\item
  \textbf{\texttt{is\_displayed}}: Determines if the page will be
  displayed.

\begin{Shaded}
\begin{Highlighting}[]
\KeywordTok{def}\NormalTok{ is\_displayed(player):}
    \ControlFlowTok{return}\NormalTok{ player.some\_condition}
\end{Highlighting}
\end{Shaded}
\item
  \textbf{\texttt{vars\_for\_template}}: Sends variables to HTML
  templates.

\begin{Shaded}
\begin{Highlighting}[]
\KeywordTok{def}\NormalTok{ vars\_for\_template(player):}
    \ControlFlowTok{return}\NormalTok{ \{}\StringTok{\textquotesingle{}variable\textquotesingle{}}\NormalTok{: player.some\_variable\}}
\end{Highlighting}
\end{Shaded}
\item
  \textbf{\texttt{get\_form\_fields}}: Specifies form fields to be
  displayed.

\begin{Shaded}
\begin{Highlighting}[]
\NormalTok{form\_model }\OperatorTok{=} \StringTok{\textquotesingle{}player\textquotesingle{}}
\NormalTok{form\_fields }\OperatorTok{=}\NormalTok{ [}\StringTok{\textquotesingle{}some\_field\textquotesingle{}}\NormalTok{]}
\end{Highlighting}
\end{Shaded}
\item
  \textbf{\texttt{before\_next\_page}}: Actions before moving to the
  next page.

\begin{Shaded}
\begin{Highlighting}[]
\KeywordTok{def}\NormalTok{ before\_next\_page(player):}
\NormalTok{    player.calculate\_something()}
\end{Highlighting}
\end{Shaded}
\item
  \textbf{\texttt{js\_vars}}: Sends variables to JavaScript in the
  template.

\begin{Shaded}
\begin{Highlighting}[]
\KeywordTok{def}\NormalTok{ js\_vars(player):}
    \ControlFlowTok{return}\NormalTok{ \{}\StringTok{\textquotesingle{}js\_variable\textquotesingle{}}\NormalTok{: player.some\_variable\}}
\end{Highlighting}
\end{Shaded}
\end{itemize}
\end{block}
\end{frame}

\begin{frame}[fragile]{Forms and User Input}
\phantomsection\label{forms-and-user-input}
\begin{itemize}
\tightlist
\item
  Storing data at the model level (in \textbf{fields})
\item
  Data validation:
\item
  Static (on the field level)
\item
  Dynamic (\texttt{\{\{field\_name\}\}\_min},
  \texttt{\{\{field\_name\}\}\_max} etc)
\item
  Getting data from user:
\item
  Defining \texttt{form\_model} and \texttt{form\_fields} at the page
  level
\item
  Showing them at the specific place of the html page using
  \texttt{\{\{formfields\}\}}
\end{itemize}
\end{frame}

\begin{frame}[fragile]{Displaying Data to Users}
\phantomsection\label{displaying-data-to-users}
\begin{itemize}
\tightlist
\item
  Showing static (\texttt{Constants)}and dynamic (from other users/same
  user) data
\item
  Showing data conditionally (\texttt{\{\{if-else\}\}} structures) and
  in arrays (\texttt{\{\{for\}\}})
\item
  Custom data formatting for a template (using
  \texttt{vars\_for\_template} and \texttt{js\_vars})
\end{itemize}
\end{frame}

\begin{frame}[fragile]{Concept of Apps in oTree}
\phantomsection\label{concept-of-apps-in-otree}
\begin{itemize}
\item
  An \texttt{app} is a set of models, pages, and templates.
\item
  Specific configuration in \texttt{settings.py} can group several apps
  together or launch the same app (or set of apps) with different
  parameters.
\item
  This can be used for between-session treatment assignment
\end{itemize}
\end{frame}

\begin{frame}[fragile]{Exercise: Creating Your First oTree Project}
\phantomsection\label{exercise-creating-your-first-otree-project}
\small

\begin{itemize}
\item
  Check if otree is installed: \texttt{otree\ version}
\item
  Create an empty project: \texttt{otree\ startproject\ YOURPROJECTNAME}

  (choose `No' when asked whether to add sample games!)
\item
  Move to a newly created folder:\texttt{cd\ YOURPROJECTNAME}
\item
  Create a new app: \texttt{otree\ startapp\ YOURAPP}
\item
  Register app in the \texttt{settings.py:}

\begin{verbatim}
    dict(
        name='public_goods',
        app_sequence=['YOURAPP'],
        num_demo_participants=3,
    ),
\end{verbatim}
\item
  Launch the server: \texttt{otree\ devserver}

  \begin{itemize}
  \tightlist
  \item
    Sometimes right after installation oTree can ask to delete the
    temporary database file first (\texttt{db.sqlite3}) - just delete
    it.
  \end{itemize}
\item
  Go to the local server in your browser: \url{https://localhost:8000}
\end{itemize}
\end{frame}

\end{document}
